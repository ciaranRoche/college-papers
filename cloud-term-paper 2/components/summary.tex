\section{Summary}
In the Kubernetes world, Operators are very much a buzz word at the moment, as its ecosystem is growing quickly with a fast adoption rate for some major players. I first heard about operators on my internship at Red Hat, unfortunately the team I worked on didn't get much exposure to them until I was leaving. That said in the single sprint where we had an operator in our backlog I got to see the potential and power of them, away from the examples I have shown in this paper. We utilized the watch function of an operator to watch for changes within a cluster and through these changes it would trigger updates to a UI. Unfortunately I didn't get to work on the operator, I worked on the UI, but regardless I gained an appreciation for operators and my interest was sparked, with that I took this paper as an opportunity to gain some experience with them.
\\It was an experience exploring a new technology and there was many challenges presented in the lack of tutorials and blogs around the subject. The sole reliance on documentation left me with an overall greater appreciation I felt. At face value I failed to see what operators offered, it was only when I got into it and nearing the end of the paper, did I see how valuable and powerful it can be to be able to inject your business logic into your automation. 
\\Seen the innovation that is happening in the community and the work going into the Operator Framework leaves me excited to see where operators will be in a years time. In my opinion I found them intimidating at the start due to the amount of boiler code. I would prefer to have some it this abstracted away, but that said I feel the boiler plate is probably needed as the complexity of an operator increases.
\\Operators bring a new level to Kubernetes, which is all ready extremely powerful. With the growth of Kubernetes and Containers in general it would be great to see a switch to treating the likes of AWS and Azure as a commodity, with out the fear of vendor locking. Allowing us to chose a cloud provider based on performance and cost and not based on services.
\\A video was made to compliment this paper showing some of the work carried out throughout. It can be found at Appendix \ref{appendix:opdemoyou}

