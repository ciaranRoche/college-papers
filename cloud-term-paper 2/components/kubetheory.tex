\section{Kubernetes}
\label{sub:kubernetes}
An understanding of Kubernetes is needed and some of the components to which Operators build on top of or extend before we continue. Kubernetes is an open-source platform for managing containerized workloads and services. It provides a container-centric management environment that orchestrates computing, networking and storage infrastructure on behalf of a user. To help put it in perspective we can think of Kubernetes as a Platform as a Service with the flexibility of Infrastructure as a Service \citep{kuberneteswhatis}.

Kubernetes uses a number of abstractions to represent a cluster, some of which are important to our understanding of Operators and will be mentioned throughout this paper.

\subsection{Overview}
To work with Kubernetes we use the Kubernetes API objects to set our desired state. A desired state consists of what applications we want to run, what container images, the number of replicas, what network and disk resources we want available etc. Normally we do this through a command-line interface, kubectl. Through the kubectl we can interact with the cluster directly, setting or modifying the desired state \citep{kuberneteswhatis}. 

Kubernetes contains a number of abstractions that represent the state of the system, these abstractions are represented by objects. It is some of these objects which Operators are built atop of or extend. 

\subsection{Custom Resources}
A resource is an endpoint in the Kubernetes API that stores a set of API objects. For example the pods resource contains a set of Pod objects. A custom resource on the other hand is one that is necessarily not available to every cluster in that it represents a customization of a particular installation. Once a custom resource is installed we can create and access its objects through the kubectl, same as how we interact with a regular resource \citep{customresources}.

\subsection{ReplicaSet}
A controller ensures that a specified pod replicas are running at any one time. A ReplicaSet is classed as a Replication Controller, these can be used independently but they are mainly used by Deployments as a mechanism to orchestrate pod creation, deletion and updates \citep{controllers}.

\subsection{Kubernetes Objects}
For brevity we have left out a lot of Kubernetes Objects and only mentioned some that need a clearer understanding when it comes to Operators. Basic Kubernetes objects include:
\begin{itemize}
    \item Pod
    \item Service
    \item Volume
    \item Namespace
\end{itemize}
Atop of basic object Kubernetes has a number of high-level abstraction called Controllers, these include:
\begin{itemize}
    \item ReplicaSet
    \item Deployment
    \item StatefulSet
    \item DaemonSet
    \item Job
\end{itemize}

\subsection{Role-Based Access Control}
A summary and a high level understanding of RBAC is needed ahead of Section \ref{sub:olm}, where we look at the Operator Lifecycle Manager, RBAC is mentioned.
\\RBAC is a method for regulating access to resources based on the roles of users. Throughout the practical elements we will be deploying roles and role bindings. The roles will connect API resources to operations, such as create, update, delete etc. The role binding then connects the roles to the users. 