\section{Summary}
\begin{figure}[!ht]
\centering
\includegraphics*[width=0.7\textwidth]{images/sum1.png}
\caption{\em Rightscale 2019 State of the Cloud Report.}
\label{img:s1}
\end{figure}

With the rise of cloud adoption by companies the need for configuration management tools is significant. Figure \ref{img:s1} shows the tools being used today, showing Ansible to be leading the poll. With this fact in mind, gives the importance of gaining experience with Ansible through this exercise. Having already completed similar exercises within the last year in manually configuring haproxy, it is easy to see the benefits configuration management has. Personally I found having the power to quickly spin up and tear down beneficial as it allowed for easier troubleshooting. During manual configuration I found it quite time consuming troubleshooting errors.
\\When comparing and looking at Terraform and Ansible, the both had a pretty low barrier to entry, being easy to upskil on. I leaned more toward Terraform, in the syntax being cleaner and easier to understand. This being just a personal opinion. Mixed with a much easier install and set up of the tool itself. Ansible on the other hand is quite a lot more mature, with that there is quite a lot of resources on line which helped in the upskilling.
\\After getting a taste of automation with this lab, along with some terraform work from my FYP, it will be quite hard to go back to manually configuring infrastructure again