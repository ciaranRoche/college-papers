\section{Summary}
For my summary, I really do not know where to begin. It has been an awesome journey this semester working with and on Knative. From a high level non technical view of the project, I can not say enough good things about it and the community behind it. Between the work for my FYP and this term paper, I have created several Google Group Threads, Github Issues, and an abundance of slack conversations with the devs. Along with one contribution to the project too. Without the community, I do not think I would of had a showcase app running to demo. I can really see the appeal for people and Open Source when they experience what I have experienced these last few weeks. 
\\Now to dig a bit deeper, Knative is super new, still in Alpha and only announced around a year ago. They put emphasis around the pronunciation of the name, being Knative (Kay nay tive) this is because before it was announced people wanted to make the 'K' silent and just pronounce it as 'native'. Why am I telling you this? because they made a big deal about the 'native' part, the main goal of the project is to make something that feels native to Kubernetes and native to cloud centric development. All I can say, in these early days is they have achieved this. As their components are custom resources we are getting new primatives to interact with Kubernetes, just like calling get pods, we can call get builds. It all feels right when working with it. 
\\It does a great deal in abstracting away unnecessary bits that are normally needed for development in Kubernetes. For example, to deploy one of my functions, say the list function to Kubernetes it required a manifest file 40 + lines long. A Knative manifest for the same deploy is just 22 lines. Around half the size. Why do I know this? I did find debugging tricky in a sense, as I would jump into a pod to see what was going on and if it would be possible to get more logs then what I was getting from the Google Console, then it would be scaled back to 0, so to give myself some freedom I deployed my list function as a normal pod to give me some time to debug and troubleshoot. 
\\Which brings me nicely to the down side of Knative, which I found was the debugging, much like the same lab with AWS there is so many moving parts, where do you begin? where is the issue? According to the docs you debug Knative the same as you debug any Kubernetes app, which now I am comfortable to do, but at the time when I was running into errors I was out of my depth. And the whole experience was a nasty one, mixed with the pressure of the this term paper and my FYP riding on over coming the problems.
\\On more downside is the initial spin up time of a pod from 0. They have it benchmarked at around 4 seconds which is way too long, they admit that. But as this project is backed by so many companies, IBM is spearheading work on getting that to sub 1 second times, and according to the devs they are close, soon we will have sub 1 second spin up from 0, which is just mind blowing. On the back of that, and something I mentioned before, is all the exciting tools which are being built, providing more abstraction and cleaner work flows to developing on Kubernetes. Only time will tell how far and how big this will go.
\\But we got to remember that this is just a tool, there is many serverless platforms out there, as it is a hot topic. So it really boils down to what tool makes the most sense for you. But one thing I do feel, all these platforms can take something from each other and learn from each other going forward. 
