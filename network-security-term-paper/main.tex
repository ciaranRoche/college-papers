\documentclass{article}

%-----------------------------------------------PACKAGES-------------------------------------------------------------%
\usepackage{graphicx} %images
\DeclareGraphicsExtensions{.pdf,.png,.jpg} % configures latex to look for the following image extensions
\usepackage{setspace} % allows for configuring the linespacing in the document
\usepackage[font=small,labelfont=bf]{caption}
\usepackage{natbib}
\usepackage[a4paper, total={6in, 8in}]{geometry}
\usepackage[explicit]{titlesec}
\usepackage[toc,page]{appendix}
\usepackage{hyperref}
\usepackage[dvipsnames]{xcolor}
\usepackage{cleveref}
\setcounter{tocdepth}{2}
\usepackage[nottoc]{tocbibind}
\usepackage[parfill]{parskip}
\usepackage[english]{babel}
\usepackage{blindtext}
\usepackage{color}
\usepackage{listings} % to use code snippets 
\usepackage{color}
\lstdefinelanguage{GoLang}{
  keywords={break, case, chan, const, continue, default, defer, else, fallthrough, for, func, go, goto, if, import, interface, map, package, range, return, select, struct, switch, type},
  keywordstyle=\color{blue}\bfseries,
  ndkeywords={class, export, exports, boolean, throw, implements, import, this},
  ndkeywordstyle=\color{darkgray}\bfseries,
  identifierstyle=\color{black},
  sensitive=false,
  comment=[l]{//},
  morecomment=[s]{/*}{*/},
  commentstyle=\color{purple}\ttfamily,
  stringstyle=\color{red}\ttfamily,
  morestring=[b]',
  morestring=[b]"
}

\lstset{
   language=GoLang,
   backgroundcolor=\color{lightgray},
   extendedchars=true,
   basicstyle=\footnotesize\ttfamily,
   showstringspaces=false,
   showspaces=false,
   numbers=left,
   numberstyle=\footnotesize,
   numbersep=9pt,
   tabsize=2,
   breaklines=true,
   showtabs=false,
   captionpos=b
}

\definecolor{lightgray}{rgb}{.9,.9,.9}
\definecolor{darkgray}{rgb}{.4,.4,.4}
\definecolor{purple}{rgb}{0.65, 0.12, 0.82}


\usepackage{hyperref}
\newcommand\myshade{85}
\colorlet{mylinkcolor}{NavyBlue}
\colorlet{mycitecolor}{YellowOrange}
\colorlet{myurlcolor}{Aquamarine}

\hypersetup{
  linkcolor  = mylinkcolor!\myshade!black,
  linktocpage=true,
  citecolor  = mycitecolor!\myshade!black,
  urlcolor   = myurlcolor!\myshade!black,
  colorlinks = true,
}

\bibliographystyle{agsm}
\setcitestyle{authoryear,open={(},close={)}}

\begin{document}
\onehalfspacing
\hypersetup{pageanchor=false}
\begin{titlepage}

\newcommand{\HRule}{\rule{\linewidth}{0.5mm}}

\center

\includegraphics{images/logo.png}\\
\textsc{\Large Project Feasibility Report}\\[0.5cm]
\textsc{\large Bachelor of Science (Hons) Applied Computing }\\[0.5cm] 


\HRule \\[0.4cm]
{ \huge \bfseries RubiX - Functions as a Container}\\[0.4cm] 
\HRule \\[1.5cm]

\Large Ciaran Roche - 20037160\\[3cm]
\large Supervisor: Dr. Rosanne Birney
\\\large Second Reader: Joe Daly

{\large \today}\\[1cm]

\vfill 

\end{titlepage}

\hypersetup{pageanchor=true}
\clearpage
\begin{center}
\begin{minipage}{\textwidth}
  
  {\scshape\large \textbf{Plagiarism Declaration}\par}
  \vspace{1cm}
  Unless otherwise stated all the work contained in this report is my own.  I also state that I have not submitted this work for any other course of study leading to an academic award.
\end{minipage}
\end{center}
\vfill % equivalent to \vspace{\fill}
\clearpage

\tableofcontents

\newpage


\newpage
\section{Introduction}
\subsection{Motivation}
\label{sub:motivation}
As seen in Figure \ref{img:strend}, serverless computing is the number one growing cloud service \citep{survey_2018}. As with many trends in software, there is no one clear view of what \gls{Serverless} is. It was first used to describe applications that fully incorporate  third-party, cloud-hosted applications and services. Another term often used to describe these type of architectures is Backend as a Service or \gls{BaaS}. Today \gls{Serverless} is the term used to describe applications where the server-side logic is written by the applications developer but unlike traditional architectures it is event-triggered, meaning that a server is not constantly running but instead events trigger the compute containers \citep{Martin.Fowler}. This is often referred to as Functions as a Service or \gls{FaaS}. A popular implementation of this is \gls{AWS Lambda}. For the purpose of this report. I will be using the \citep{Martin.Fowler} definition of \gls{Serverless}.

\begin{figure}[!ht]
\centering
\includegraphics*[width=1\textwidth]{images/a1.png}
\caption{\em Top Growing Cloud Services -- \cite{survey_2018}}
\label{img:strend}
\end{figure}

One of the main benefits and motivation for the project is that it reduces operational cost, as one only pays for the compute power they need as opposed to a constant running server instance. Another aspect of the operational costs stem from the development as developers can focus on business logic of their application over building the infrastructure around their application \citep{Martin.Fowler}. 

The \gls{Cloud Foundry Foundation}\footnote{https://www.cloudfoundry.org -- a multi-cloud application platform.} conducted a global survey of their users and found that 22\% are already using \gls{Serverless} technology \citep{foundry_2018}. And with almost half their users evaluating the technology. This kind of trend is supported by research conducted by Cloudability \citep{report:cloudability}. This suggests that the area of \gls{Serverless} computing is growing rapidly.

Further backing the growth of \gls{Serverless}, Google announced the launch of their \gls{Knative}\footnote{https://cloud.google.com/knative/ -- middleware components for container-based applications.} project which is based on the \gls{Kubernetes}\footnote{https://kubernetes.io -- container-orchestration system for automating deployment.} engine and developed by Google, Pivotal, IBM, Red Hat and SAP. Knative provides a suite of developer-focused middleware tools for building, deploying and managing serverless applications on the \gls{Kubernetes} engine \citep{bryant_2018}.

The adoption rate of \gls{Serverless} technologies, mixed with a personal interest in the migration of conventional monolithic applications to microservice \gls{Serverless} applications, led me to explore problems around \gls{Serverless} that could be solved as a topic for my Final Year Project.

\subsection{Problem Statement}
\label{sub:problem}
 There are a number of identified problems with \gls{Serverless} technologies. A report by DigitalOcean highlighted that the biggest challenge faced by \gls{Serverless} was the ability to monitor and debug \citep{digitalocean_2018}. This was shown to be down to the layers of abstraction by particular vendors and their \gls{Serverless} services.
\\While \gls{Knative} looks after the orchestration of a \gls{Serverless} application within \gls{Kubernetes} seamlessly, thus reducing the number of layers of abstraction, it is up to the developer to build, configure and package their functions in a container. This pattern goes against one of the main benefits of \gls{Serverless} in that a developer can focus their time on the business logic over the configuration of the infrastructure thus reducing operational costs.
\\This project will look to solve both problems outlined above and will be discussed in detail in Section \ref{sub:aims}.

\subsection{Aims and Objectives}
\label{sub:aims}
With the aim to solve the problems outlined in Section \ref{sub:problem} the overall goal is to provide an \gls{Open-Source} tool to allow software engineers to easily create \gls{FaaC} to be utilised by \gls{Knative}. 

\begin{itemize}
    \item [\textbf{AO1}] - Provide support for building \gls{FaaC} in multiple languages. This will reduce the cost of entry, as developers can use a language that best fits their use case or expertise.
    \item [\textbf{AO2}] - The ability to easily debug functions. Solving the main identified problem with \gls{Serverless}, providing the means for easy debugging will help increase adoption rate.
    \item [\textbf{AO3}] - Confidence in a Function as a Container being as secure as possible. By removing layers of abstraction the developers can have confidence in their containers being up to date and as secure as possible.
    \item [\textbf{AO4}] - The project to be transparent and developed in the open. This will allow for the community to mould and adapt the project to best fit their needs, overall increasing the quality and the uptake of the project.
    \item [\textbf{AO5}] - The means to control the life cycle of a Function as a Container. This will increase the speed of local development by providing the developer with the tools needed to fit their development needs, thus reducing the need to switch between tools.
\end{itemize}

The project will utilise common patterns that would be native to a software engineer already working in the space of \gls{Container}s and \gls{Kubernetes}. It will consist of the following:

\begin{itemize}
    \item \textbf{CLI:} An easily extensible command line interface following the same work flow such as tools like \gls{Docker} CLI and \gls{Kubernetes} Kubectl.
    \item \textbf{SDK's:} A suite of \gls{SDK}'s to support the creation of Functions as a Container in multiple programming paradigms.
    \item \textbf{Showcase Application:} To demonstrate a modern \gls{Serverless} application running in the \gls{Kubernetes} engine.
\end{itemize}

\subsection{Contributions}
The end result of this project will be a Framework which provides a suite of \gls{SDK}'s and a \gls{CLI} that will reduce the barrier of entry to \gls{Serverless} development. The project will be developed in the open with the goal to build a community that will allow the development be guided by the needs of the \gls{Serverless} community as a whole. On the back of the project a number of applications to talk at conferences will be submitted, with a local tech talk already confirmed for April 2019.


\subsection{Outline}
This report is broken into seven sections. This section, the first, is an overall introduction covering motives and the problem to be solved along with contributions that I intend to make. In section two, the work carried out in Semester One is summarised. The third section looks at  the methodologies this project follows. Section four examines the technological choices made. Section five outlines the tools used to aid development of the project. The system architecture, requirements and user stories make up the design of the project, which is captured in Section six. Section seven discusses the implementation of the project,  consisting of Scrum Ceremonies along with personal reflections on each sprint. Finally section eight reviews the project direction, reviewing the work completed. The learning outcomes and a final personal reflection are also included in this section.
\clearpage 
This project is sponsored by Red Hat, acting as stakeholders in the project. As stakeholders Red Hat fulfill the role of product owner ensuring the development of this product meets the requirements set to solve the problem stated in Section \ref{sub:problem}. This is not a paid or monetary sponsorship.

This project is supervised by Dr. Rosanne Birney.

\newpage
\section{Description}
\label{sub:description}

\subsection{What is a Container}
\label{sub:container}
Containers are a modern version of capabilities that have been available in Unix operating system Solaris for decades, in that you can run processes in isolation to the rest of the operating system \citep{oracle_solaris}. When it comes to applications, traditionally one would have a virtual machine which would house the application. This would mean the applications processes would run in isolation to the operating system on its host but it does this with the cost of a massive over head in that of the entire virtual machines operating system consumers memory and resources \citep{yegulalp_2018}.

Containers are a means that looks at how to run applications and processes in isolation, and they do this by giving applications a partitioned segment of the hosts operating system. This is managed by the container run time environment, while there is many run time environments that support OCI containers for the majority of this paper the focus will be on Dockers run time called libcontainer \citep{docker_2018}.

\begin{figure}[!ht]
\centering
\includegraphics*[width=\textwidth]{images/docker-view.png}
\caption{\em Docker Container vs Virtual Machine (image from Docker)}
\label{img:docker_vm}
\end{figure}

Figure \ref{img:docker_vm} paints a clear picture of the difference between a conventional system of applications running on a virtual machines (VMs) and applications running in containers. As can be seen from the figure, VMs are an abstraction of the physical hardware. The hypervisor allows multiple VM's run on a single hosts operating system. Each VM has is own operating system along with all the binaries, libraries and packages needed for the applications. 

Where as in the Figure \ref{img:docker_vm} it can be seen that containers are more of an abstraction at the application layer. As with VMs multiple containers can run on the same machine, but unlike VMs, containers share the OS kernel, all running in isolation based on a given space \citep{docker_2018}. 

An understanding of cgroups and namespaces is needed to understand how containers share the OS kernel. It is these features that build the boundaries between containers and processes running on a host \citep{yegulalp_2018}. 

\subsection{Control Groups}
\label{sub:cgroups}

Control Groups or Cgroups for short allow for the allocation of resources such as CPU time, system memory, network bandwidth or a combination of the resources. cgroups allow for the fine-grained control over allocating, prioritizing, denying and managing system resources. Management of these resources increase overall efficiency.

It must be noted that all processes on a Linux based system are child processes of a common parent. The \textit{init} process, which is executed at boot time by the kernel starts all other processes. Thus the Linux process model is a single hierarchical tree. 

Cgroups share similarities in that they are hierarchical and child cgroups inherit some attributes from their parent cgroup. This allows for simultaneous crgoups on a system. To paint a clearer picture if we describe the Linux process model as a single tree of processes, then the cgroup model is made up of one or many unconnected trees of processes \citep{cgroups_2018}.

\subsection{Namespaces}

Namespaces act similar to cgroups in that they deal with resource isolation, but unlike cgroups which deals with a number processes, Namespaces only isolate a single process. Linux Namespaces wrap a global system resource and makes it appear to the processes within its namespace that the global system resource is dedicated to that process \citep{namespaces}.

Furthermore these namespaces are broken into six different types, each namespace wraps a particular global resource. Due to this the overall nature of namespaces is to support the implementation of containers \citep{kerrisk_2013}.

\subsubsection{UTS namespaces}
In the space of containers, the UTS namespaces allow each container to have its own hostname and NIS domain name. This is used for the initialization and config scripts that take action based on the namespace. The term UTS gets its name from the 'Unix Time-Sharing System" \citep{kerrisk_2013}

\subsubsection{IPC namespaces}
IPC namespaces allow for its own interprocess comminication resource. In other words it provides shared memory spaces for accelerated communication (POSIX message queue filesystem) \citep{kerrisk_2013}.

\subsubsection{PID namespaces}
PID namespaces isolate the process ID number space. To put differently processes in different PID namespaces can have the same PID. The main benefit to this is that containers can be migrated between hosts and keep the same process IDs for all processes inside the container \citep{kerrisk_2013}.

\subsubsection{Network namespaces}
Network namespaces provide isolation network resources in that each network namespace has its own network devices, IP routing tables, port numbers, IP addresses etc. Network namespaces make containers extremely powerful from a network perspective. In that you could have multiple containers on the same host all bound to port 80 in their own network namespace \citep{kerrisk_2013}. Given that each container has its own virtual networking, tools like Docker Swarm and Kubernetes utilize this and provide suitable networking rules on the host.

\subsubsection{User namespaces}
\label{sub:username}
User namespaces isolate the user and group ID number spaces. This means a process's user and group ID's can be different inside and outside of a user namespace. A typical use case would be a process would have full root privileges for operations within the user namespace but is unprivileged for operation outside the namespace \citep{kerrisk_2013}.

\subsubsection{Mount namespaces}
Mount namespaces isolate filesystem mount points seen by a group of processes. This allows for processes in different mount namepaces to have different views of a filesystem hierarchy \citep{kerrisk_2013}.

\subsection{Capabilities}
\label{sub:capabilities}
As Linux capabilities are mention during the practical element in Section \ref{sub:escape} it is worth providing some theory background here on what Linux capabilities are.

Traditional UNIX systems has two categories of processes, privileged and unprivileged. Privileged processes are identified as user ID 0, or also known as superuser or root. Where as unprivileged processes have a UID of nonzero \citep{linux-manual}.

\subsection{Open Container Initiative}
The Open Container Initiative was established in June 2015 by Docker, CoreOS and many other leaders in the container industry. The purpose of the initiative was to create open industry standards around container formats and runtime. Currently the OCI have two specifications, the runtime specification and the image specification \citep{oci}. 

The purpose of these specifications is to ensure that all OCI compliant technologies are interoperable with each other, thus reducing the likely hood of a major company take over controlling the entire Container market.

Due to the interoperablity of OCI compliant containers the rest of this paper will focus on Docker, but all topics covered are applicable to not just Docker but all OCI compliant technologies.

\subsection{Docker}
\label{sub:dock}
Dockers underlying architecture can be described as a client and server (docker daemon) based. This is shown in Figure \ref{img:docker-arch}, the server receives requests from the client through a RESTful API. The API along with a command line client are shipped by Docker. Due to make up of the architecture it allows for the docker daemon and client to run on the same machine or alternatively the client can connect to a remote docker daemon \citep{rad_bhatti_ahmadi_18}. It is the Docker daemon which orchestrates and manages all containers and images on the host where as the client is the control of the daemon.

\begin{figure}[!ht]
\centering
\includegraphics*[width=0.7\textwidth]{images/docker-arch.png}
\caption{\em Docker Architecture (image from Docker)}
\label{img:docker-arch}
\end{figure}

Also depicted in Figure \ref{img:docker-arch} is a Docker Image. A Docker image is essentially an immutable snapshot of a container. Running a Docker build command creates images. It is from this created image that a container is created by the Docker run command. Generally images are stored on a registry and will be demonstrated in Section \ref{sub:practical}. 

Docker Hub would be a main source for storing Docker images, and it itself is a huge security risk, as the Hub allows anybody to push their images to be used by themselves or other people. A study carried out in 2016 showed out of 352,416 community images tested on Docker Hub the worst image contained almost 1,800 vulnerabilities, with of 3,892 official images tested the worst image contained almost 800 vulnerabilities \citep{colyer_2017}. The topic of Docker image authenticity is covered in Section \ref{sub:authent}.

\subsection{Description Summary}
So we have looked at what a container is and compared it to a traditional virtual machine. It was discussed how containers use control groups and namespaces to isolate processes from those of other containers and the host machine. We outlined what the Open Container Initiative is and discussed the architecture of Docker. Backed by the knowledge learned, we will look at a number of use cases based on a number of security services in Section \ref{sub:practical}.







\newpage
\section{Work Carried Out}
\label{sub:practical}
The work carried out will look at container security threats with an emphasis on Docker. Each element will be divided into a number of sections. Out lining the threat description, what the attack could be and why it affects Docker. We will then look at the best practices in Docker to prevent this kind of security threats. Finally look at a reproducible proof of concept to showcase the threat.

Each element below will fall into one of the following security services, authentication, access control, data confidentiality, data integrity, non-repudiation and availability. Authentication is the correct identification of an entity or a source of data. Access control is who can access what and in what way. Data confidentiality is the non-disclosure to external parties. Data integrity is the correctness of data. Non-repudiation is the proof that communication actually took place. Finally availability is to ensure that a system is available when it is required. 

Label \ref{tab:prac-sec} categorizes each element to a particular security service. 

\begin{table}[!ht]
\begin{center}
\begin{tabular}{ ||c|c|| } 
 \hline
 \textbf{Security Service} & \textbf{Section} \\ 
 \hline
 \hline
 Authentication & \ref{sub:escape}, \ref{sub:authent}, \ref{sub:cred-secre} \\
 Access Control & \ref{sub:hostd} \\
 Data Confidentiality & \ref{sub:cred-secre}, \ref{sub:static-image} \\
 Data Integrity & \ref{sub:escape}, \ref{sub:authent} \\
 Non-Repudiation & \ref{sub:cred-secre} \\
 Availability & \ref{sub:abuse} \\
 \hline
 \hline
\end{tabular}
\caption{\em Security Service Elements}
\label{tab:prac-sec}
\end{center}
\end{table}

\subsection{Docker Host and Kernel Security}
\label{sub:hostd}
\subsubsection{Description}
If a host system is compromised the container isolation outlined in Section \ref{sub:container} will not make any difference. From Figure \ref{img:docker_vm} we saw that container run on top of the host kernel. This gives the advantages of being efficient over conventional virtual machines, but from a security perspective it can be seen as a risk.
\subsubsection{Best Practices}
Docker supply a bench auditing tool which will be discussed and demonstrated in Section \ref{sub:auditDocker} with a number of Open Source auditing tools available will also be discussed and demonstrated in Section \ref{sub:auditSource}. These tools as a whole check the configuration for best practices. 

It is often the case with building containers to simply fire and forget. This is due the choice of base image which was discussed in Section \ref{sub:dock} or even how Docker blocks certain behaviour on a container. It is often good practice to enforce your own Mandatory Access Control. This will resolve any unwanted operations, these controls can be placed at both the host and on the container at the kernel level and can be utilized from many tools including Seccomp, SELinux, and AppArmor. For the purpose of this paper we will look at using Seccomp. The profile used will be from Moby which is a Docker backed project and can be found at Appendix \ref{appendix:seccomp}. A simple rule of thumb for building containers if I don't need why have it there.

\subsubsection{Proof of Concept}
Seccomp can be thought of as a firewall for the kernel call interface. Regardless of this some calls are already blocked by default Docker profile:
\begin{figure}[!ht]
\centering
\includegraphics*[width=\textwidth]{images/term1.png}
\caption{\em Kernel Security Demo 1}
\label{img:demo1}
\end{figure}

As we can see from Figure \ref{img:demo1} we have run a standard docker alpine image and started a bash session in the container. When we ran \textit{whoami} we can see that we have \textit{root} access. As can be seen Docker as standard blocks the \textit{mount} call but does not block the \textit{chmod} call. Allowing us to modify permissions on files and folders.

This can be rectified by applying our own Seccomp profile to the container. If we check Appendix \ref{appendix:seccomp} we can see a list of syscalls on line 52 which are whitelisted. If we remove \textit{chmod} from the list and run our container with the \textit{--security-opt} flag we can see we have successfully blocked the use of the \textit{chmod} call.

\begin{figure}[!ht]
\centering
\includegraphics*[width=\textwidth]{images/term2.png}
\caption{\em Kernel Security Demo 2}
\label{img:demo2}
\end{figure}
\newpage
\subsection{Docker Container Escape}
\label{sub:escape}
\subsubsection{Description}
To escape or breakout from a container is the term used to when container isolation checks have been bypassed. Allowing an attacker to gain access to sensitive information from the host or gaining additional privileges. One of the most popular escapes in containers is the Dirty COW exploit CVE-2016-5195 \citep{cve-2016-5195}. The Dirty Cow is a race condition found in the way Linux kernel's memory subsystem handles the copy-on-write breakage of private read-only memory mapping. Allowing an unprivileged user to gain write access to read-only memory mappings. As this exploit is known it is often patched but the CVE affects all Linux kernels, thus enforcing the importance of maintaining an updated container.

Away from the exploit Docker daemon runs as root by default. It is good practice to create a user-level namespace or simply drop some of the container root capabilities.

\subsubsection{Best Practices}
As mentioned previous, if I don't need it why do I have it. So dropping capabilities that are not required by the software within the container is a must. For example, \textit{CAP\_SYS\_ADMIN} grants a wide range of root level permissions. Micheal Kerrisk is quoted saying "CAP\_SYS\_ADMIN is the new root \citep{kerrisk_2012}. Dropping these kind of capabilities is a must for a secure container.

As discussed in Section \ref{sub:capabilities} docker container run as UID 0 to avoid this a user should be created in an isolated user namespace limiting the privileges over that of the host regular user.

If a you have to run a privileged container, always check that it is from a trusted source which is discussed in Section \ref{sub:authent}.

Always check your mount points from the host. For instance the Docker socket /var/run/docker.sock, /proc, /dev are special mounts that perform the containers core functionality. Insure you understand the why and the how to limit processes from gaining access to this privileged information. Sometimes it is enough just to expose the file system with read-only privileges. 
\subsubsection{Proof of Concept}
\begin{figure}[!ht]
\centering
\includegraphics*[width=0.9\textwidth]{images/term3.png}
\caption{\em Container Escape Demo 1}
\label{img:demo3}
\end{figure}
Docker by default allows the root account to create device files, you should restrict this if your application does not require it. As you can see from Figure \ref{img:demo3} we run an alpine image and can easily create a device file. Where as if we launch an alpine image with the flag \textit{--cap-drop=MKNOD} the operation is not permitted.
\begin{figure}[!ht]
\centering
\includegraphics*[width=0.9\textwidth]{images/term4.png}
\caption{\em Container Escape Demo 2}
\label{img:demo4}
\end{figure}
A root user will override any file permissions by default. It is good practise to restrict this in containers using different users which may contain mounts with sensitive data. As can be seen from Figure \ref{img:demo4} we launch a ubuntu image. Create a new user and user home directory. We create a file and pipe some text to it. We set the permissons to \textit{600} so that only the owner can read and write to the file. We exit back to root user and as can be seen root is able to \textit{cat} the file. 

Whereas when we launch a ubuntu image with the flag \textit{--cap-drop=DAC\_OVERRIDE} if we complete the same steps the root user is denied from reading the file.

\begin{figure}[!ht]
\centering
\includegraphics*[width=0.9\textwidth]{images/term6.png}
\caption{\em Container Escape Demo 3}
\label{img:demo6}
\end{figure}

\textit{Cap-drop} is an extremely powerful command, as if we refer to Appendix \ref{appendix:capab} we can see a number of Linux capabilities which are allowed by default and need to be manually dropped. For instance \textit{NET\_RAW} is the capability to use RAW and PACKET sockets, this is common method for security scanners and malware tools. 



As you can see from Figure \ref{img:demo6} we launch an nmap container with no problems, if a container has been compromised an attacker first port of call may be to see what ports are open this is a serious vulnerability in the case if our application does not need the use of RAW and PACKET sockets, why have it as a capability in our container. Where as using the \textit{CAP-DROP} flag we can also see from Figure \ref{img:demo6} that the operation is not permitted.

\newpage
\subsection{Docker Image Authenticity}
\label{sub:authent}
\subsubsection{Description}
Docker images are littered across repositories on the Internet. But if you are pulling images without using any trust and authenticity you could be running any kind of software on your host devices. A number of questions you should ask yourself before pulling an image.
\begin{itemize}
    \item Where did the image come from?
    \item Do I trust the creator? What are their security policies?
    \item Is there any means to cryptographically prove that the author is who they say they are?
    \item How do I know no one has tampered with the image after I have pulled it
\end{itemize}
Docker allows you to pull and run any image from any source as standard. So even if you are using your own images you need to ensure no one else tampers with them. It often boils down to a Public Key chain of trust.
\subsubsection{Best Practices}
As with any piece of software it comes down to common sense, never run something that you do not explicitly trust the sources. To err on the side of caution it is always worth enforcing mandatory signature verification for any image that is to be pulled and or run on a host.
\subsubsection{Proof of Concept}
For this proof of concept it will show how to enable public key chain of trust on images.

If we refer to Figure \ref{img:demo8} there is a lot going on there, so we shall break it down. We have created a project called \textit{my-example} in it we have a \textit{Dockerfile}. This \textit{Dockerfile} just pulls an alpine image. Next we build this image an call it \textit{muldoon/alpineunsigned} muldoon being my own Docker Hub user name.

Next we login to Docker using our Docker Hub credentials. Now we can push our unsigned image to our Docker Hub. Finally if we export \textit{DOCKER\_CONTENT\_TRUST} and equal it to 1. This enables Docker trust enforcement. And as can be seen from Figure \ref{img:demo8} if we try to pull the image we get trust error.
\begin{figure}[!ht]
\centering
\includegraphics*[width=\textwidth]{images/term8.png}
\caption{\em Docker Image Authenticity 1}
\label{img:demo8}
\end{figure}
\newpage
Now if we build another image this time using the flag \textit{--disable-content-trust} and set it to false it will build the container and sign it by default. This can be seen in Figure \ref{img:demo9}
\begin{figure}[!ht]
\centering
\includegraphics*[width=\textwidth]{images/term9.png}
\caption{\em Docker Image Authenticity 2}
\label{img:demo9}
\end{figure}

With our newly signed container built we can push the image to our Docker Hub. This time we get asked to create a two passphrases. One for our root key and the other for the repository. The root key or offline key as it is also known is only needed for the creation of new repositories associated with the account. And the repository key is used for the image we just pushed so that we can identify it. This can be seen in Figure \ref{img:demo10}. With the image pushed to our Docker Hub it can be seen that we can pull the image with no trust errors.
\begin{figure}[!ht]
\centering
\includegraphics*[width=\textwidth]{images/term10.png}
\caption{\em Docker Image Authenticity 3}
\label{img:demo10}
\end{figure}

\subsection{Docker Resource Abuse}
\label{sub:abuse}
\subsubsection{Description}
As discussed in Section \ref{sub:container}, containers are lightweight in comparison to Virtual Machines, this means we can deploy multiple containers on our hardware. But with lots of entities comes the fight for resources. Be it bugs in the applications or deliberate attacks a Denial of Service can easily happen consuming all the resources on a host leaving none for the containers. 
\subsubsection{Best Practices}
Limits to resources are in fact disabled by default, so configuring of resources before deployment is a must. One method to limit resources is to utilize the resource limitation features that come with the Linux kernel. Another method is to stress test containers as part of the CI/CD cycles. Load testing is crutial when it comes to knowing the physical limits to the operations of containers. And finally make use of tooks for monitoring and alerting when resources are low.
\subsubsection{Proof of Concept}
For this proof of concept we will use cgroups that was discussed in Section \ref{sub:cgroups} to limit resources to a container. We can do this simply by running the command seen in Figure \ref{img:demo11}. As we want to test the limitations of 2G memory has been set we need to install \textit{stress} tool on the machine, so first need to run an \textit{apt-get update}
\begin{figure}[!ht]
\centering
\includegraphics*[width=\textwidth]{images/term11.png}
\caption{\em Docker Resource Abuse 1}
\label{img:demo11}
\end{figure}

With our ubuntu container updated, we can run the command seen in Figure \ref{img:demo12} to install \textit{stress}.
\begin{figure}[!ht]
\centering
\includegraphics*[width=\textwidth]{images/term12.png}
\caption{\em Docker Resource Abuse 2}
\label{img:demo12}
\end{figure}

Once stress is installed we run the command seen in Figure \ref{img:demo13} to stress test our container. 
\begin{figure}[!ht]
\centering
\includegraphics*[width=\textwidth]{images/term14.png}
\caption{\em Docker Resource Abuse 3}
\label{img:demo13}
\end{figure}

As we have set the limits to 2GB main memory, with 3GB in total to include \textit{swap}. Swapping is a technique that allows a computer to execute programs and manipulate files that are larger then main memory \citep{centos}. We can see from Figure \ref{img:demo13} that we are stressing the container with over 8GB of memory usage. As can be seen this fails. If we run \textit{Docker Stats} in a separate terminal on the host machine, we can see the container usage spike before the process is killed. As the process is looking for more resources then the cgroup will allow the process is killed.



\subsection{Docker Vulnerabilities in Static Images}
\label{sub:static-image}
\subsubsection{Description}
As containers as essentially isolated, if they are acting as expected it is easy to forget the underlying architectures and dependence's. As vulnerabilities are found daily it is important to have the appropriate measures in place to test containers to ensure any security flaws are patched and updated as soon as possible.
\subsubsection{Best Practices}
A simple but not fully effective solution would be to rebuild and update images periodically to ensure the latest patches are applied. This does have the over head of re-testing your containers to ensure no patch's have broke the application. 

While live-patching containers is considered a bad practice, Docker and tools like Kubernetes often provide the means to roll out updates without disrupting uptime of the application.

As said many times through out this paper, if you do not need it why have it. This applies here as keeping a container simple and minimal will reduce the attack surface and also reduce the need for frequent updates.

The number one approach is to provide some sort of vulnerability scanner. This will be looked at in detail in Section \ref{sub:auditSource} and Section \ref{sub:auditDocker}. These tools check containers for all known vulnerabilities and can be incorporated in CI/CD.
\subsubsection{Proof of Concept}
As this paper looks at auditing tool for vulnerabilities in images in a later section, we will look at using a different registry service to Docker hub. 

CoreOS Quay repository uses the open source security image scanner Clair. Despite being aimed for commercial use a lot of Quay's services are free to use. With an account set up on Quay.io it is easy to incorporate it into a work flow over using the conventional \textit{Docker push} which we saw earlier in the paper. 

As can be seen from Figure \ref{img:demo16} we have an image called \textit{muldoon/test-server:latest} running. We just need an extra step over directly pushing it to a registry as we do with Docker Hub, this time we need to commit it to a repository that we have created on Quay. In this case the repository that was created was called \textit{quay.io/ciaranroche/test-image}. The steps in creating the repository have been left out for brevity. 

With the image commit to the repository it is only a matter of pushing to the repository. A similar pattern seen from the likes of GitHub.
\begin{figure}[!ht]
\centering
\includegraphics*[width=\textwidth]{images/term16.png}
\caption{\em Docker Vulnerabilities in Static Images 1}
\label{img:demo16}
\end{figure}

Once the image has been pushed into the repository we can navigate to the Quay account, and from there inspect the image security scan. As can be seen from Image \ref{img:demo17} no known vulnerabilities have been found. If they had been known vulnerabilities we would receive a link to the CVE and also any upstream patched package versions.
\begin{figure}[!ht]
\centering
\includegraphics*[width=\textwidth]{images/term17.png}
\caption{\em Docker Vulnerabilities in Static Images 2}
\label{img:demo17}
\end{figure}

\subsection{Docker Credentials and Secrets}
\label{sub:cred-secre}
\subsubsection{Description}
Due to the dynamic nature of containers in that you do not just set up a server so to speak, containers are constantly created, destroyed and or updated. Due to this nature a secure process is needed to share sensitive info, be it user password hashes, encryption keys etc.
\subsubsection{Best Practices}
A common practice is storing secrets in environment variables, this is very insecure. Instead use a Docker credentials management software to manage your secrets. As a general rule of thumb never leave credentials and or secrets in a container, an IBM report can be quoted in saying "it is the same as leaving a jail cell's keys inside the jail cell" \citep{ibm}.
\subsubsection{Proof of Concept}
Figure \ref{img:demo18} shows how to capture environment variables in a docker container. We launch an alpine image with a password. Once on a bash instance on the container if we simply run \textit{env | grep pass} we can see the password is stored in plan text inside the container.
\begin{figure}[!ht]
\centering
\includegraphics*[width=\textwidth]{images/term18.png}
\caption{\em Docker Credentials and Secrets}
\label{img:demo18}
\end{figure}

With the growth in popularity of orchestration systems such as Docker Swarm or Kubernetes. They all provide a basic secret management system. For this I switch over to a Ubuntu Virtual Machine to avoid an configuration errors with my own personal machine. From Figure \ref{img:demo19} we launch docker and create a \textit{secret.txt} file and pipe in a secret.
\begin{figure}[!ht]
\centering
\includegraphics*[width=\textwidth]{images/term19.png}
\caption{\em Docker Credentials and Secrets 2}
\label{img:demo19}
\end{figure}

With a base secret file created on our host Virtual Machine, we launch Docker Swarm, and assign an IP address to be advertised, this is crucial in initializing Docker Swarm as the address is advertised to all members of the swarm and allows member API access and overlay networking, it will be used under the hood for us when we retrive the key later on. The initalize command can be seen in Figure \ref{img:demo20}
\begin{figure}[!ht]
\centering
\includegraphics*[width=\textwidth]{images/term20.png}
\caption{\em Docker Credentials and Secrets 3}
\label{img:demo20}
\end{figure}

Next we need to create a secret this can be seen in Figure \ref{img:demo21} where we create a secret called somesecret and specify the \textit{secret.txt} we created earlier.
\begin{figure}[!ht]
\centering
\includegraphics*[width=\textwidth]{images/term21.png}
\caption{\em Docker Credentials and Secrets 4}
\label{img:demo21}
\end{figure}

With the secret created and our swarm initalised we now create an nginx container and specify the created secret as can be seen in Figure \ref{img:demo23}.
\begin{figure}[!ht]
\centering
\includegraphics*[width=\textwidth]{images/term23.png}
\caption{\em Docker Credentials and Secrets 5}
\label{img:demo23}
\end{figure}

Finally if we log into the nginx container we can retrieve the secret seen in Figure \ref{img:demo24}. This is been retrived from Docker Swarm and is not stored in the container. If we run \textit{ls -l} we can see only root can read the secret with no other access.
\begin{figure}[!ht]
\centering
\includegraphics*[width=\textwidth]{images/term24.png}
\caption{\em Docker Credentials and Secrets 6}
\label{img:demo24}
\end{figure}

It has to be noted this is at a very minimum of security. But it does mean that secrets are stored properly and you can rotate and revoke them as you wish from Docker Swarm.
\newpage
\subsection{Container Auditing}
\subsubsection{Docker Bench Audit Tool}
\label{sub:auditDocker}
The Docker Bench for Security is a docker container that runs a script to check for common best-practices around deploying Docker containers. It has to be noted that the container is run with a lot of privileges for instance it shares the host's filesystem along with pid and network namespaces. For this case you must insure you trust the container and its origins before running it on a host system \citep{github_docker_2018}. 

For this test I ran multiple containers on a Ubuntu Virtual Machine. These images where pulled from Docker Hub and where commonly used, Ubuntu images, Alpine images, Nginx images, MongoDB images and SQL images. Figure \ref{img:demo25} shows the initial output. Overall no containers failed a test but we did receive a lot of warnings, many of which where outlined already in Section \ref{sub:practical}.

\begin{figure}[!ht]
\centering
\includegraphics*[width=0.9\textwidth]{images/term25.png}
\caption{\em Docker Benchmark Audit Tool}
\label{img:demo25}
\end{figure}

If you note the first Warning message in Figure \ref{img:demo25}, Ensure a separate partition for containers has been created. As this tool has been inspired by the CIS Docker Comminity Edition Benchmark Guide, you can refer to the guide to find the appropriate action this can be seen in Appendix \ref{appendix:cis} in relation to our first warning.
\newpage
\subsubsection{Open Source Audit Tool}
\label{sub:auditSource}
The majority of Docker Auditing tools, including the official tool used above in Section \ref{sub:auditDocker} are Open Source. So to give some comparison we are going to take another Open Source tool, DockScan. This tool scans containers for security issues and vulnerabilities. It can be easily installed as it is written in Ruby, we can use \textit{gem install} which can be seen in Figure \ref{img:demo28}.
\begin{figure}[!ht]
\centering
\includegraphics*[width=\textwidth]{images/term28.png}
\caption{\em Open Source Audit Tool}
\label{img:demo28}
\end{figure}

With the tool installed we launch an NGinx container locally and begin our scan this can be seen in Figure \ref{img:demo26}. As can be seen we get a human readable output outlining an flaws or vulnerabilities. In this case we can see that many of the recommendations have already been covered throughout Section \ref{sub:practical}.
\begin{figure}[!ht]
\centering
\includegraphics*[width=\textwidth]{images/term26.png}
\caption{\em Open Source Audit Tool}
\label{img:demo26}
\end{figure}
\newpage
\subsubsection{Container Auditing Summary}
As mentioned previously companies are moving away from giant monolithic applications to more microservice architectures, and with that comes the need and want to rapidly roll out updates and deployments \citep{neuvector_2018}. Implementation of a CI/CD pipeline should be a starting point for checking containers for vulnerabilities.
The examples in this Section, show two tools that can be used for auditing containers. It should be noted that these tools can be easily incorporated into a CI/CD work flow. As a good practice you should implement such tools, to insure the integrity of your containers.

\newpage
\section{Summary}

\subsection{Project Direction}
Currently the project has succeeded in delivering an MVP to the Product Owners. With the completion of the MVP I feel the project has successfully solved the problem outlined in Section \ref{sub:problem}, in providing a means to quickly and efficiently develop applications within Knative. The current state of the Project gives users an open door to explore what the Knative project has to offer, with minimal upskilling needed, giving more focus to Knative and its features.
\subsubsection{Future Development}
I feel the project currently solves the initial outlined problem, while there is work to be completed, which is documented and logged on RubiX backlog in Trello. With exposure and experience to the Kubernetes and Knative ecosystem I have discovered many new demands and needs for developers, this is further backed by the release of Quarkus\footnote{https://quarkus.io -- A Kubernetes Native Java stack.}. A high level view of Quarkus, is a container-centric Java Stack with the main goal to spin up a Java container in sub second times. 
\\The current state of RubiX provides the ground work to develop upon. Work that continues will solely focus on a single SDK, with the core goal of container first, developing near instant scale up and high density memory utilization in Kubernetes. As current trends in development within the Knative project are skewed toward scaling from 0 in sub second times. The container itself is an individual cog in this process, and development in this area is extremely valuable to the Knative community. RubiX is the perfect candidate to build upon, and this is work which I will begin on after the college term.
\subsection{Review}
\subsubsection{Core Goals}
The project delivered on all its core goals set out by the stakeholders.
\begin{itemize}
    \item [\textbf{AO1}] - Completed SDK's in the following paradigms -- TypeScript, GoLang, Python, Java and Haskell
    \item [\textbf{AO2}] - Each SDK handling and catching errors, with the ability for logging.
    \item [\textbf{AO3}] - Dockerfiles following the best practices outlined by Docker and the OCI.
    \item [\textbf{AO4}] - The project was developed in the open under the GitHub org RubixFunctions.
    \item [\textbf{AO5}] - A CLI tool developed to abstract common usages and workflows through the development of Functions as a Container.
\end{itemize}
The project deliverables are as follows:
\begin{itemize}
    \item JavaScript SDK
    \item GoLang SDK
    \item Python SDK
    \item Java SDK
    \item Haskell SDK
    \item Development CLI
    \item Knative Terraform
    \item RubiX/Knative Showcase Application
    \item RubiX Documentation
    \item RubiX HelloWorld Samples
\end{itemize}
\textbf{Contributions}
\\As Knative is an Open Source project, it gave the opportunity to contribute back to it over the course of the project. It is often a misconception that a contribution to an Open Source project has to be in the terms of code. But a contribution comes in many forms, with the goal of improving the project and developing something special and community driven.
That said over the course of my Project it resulted in the following:
\begin{itemize}
    \item One Pull Request to a Knative Repository.
    \item One Github Issue in a Knative Repository.
    \item Two authored Knative Google Group Threads.
    \item Responded to multiple Knative Google Group Threads.
    \item Partook in Slack Discussions.
\end{itemize}

\newpage
\subsection{Learning Outcomes}
This project produced multiple Learning Outcomes, ranging from Technical and Non-Technical.
\subsubsection{Technical Learning Outcomes}
\begin{itemize}
    \item Gained valuable insight into the containerization and orchestration of applications
    \item Developed an appreciation for quick prototyping and iterating within the development process, as is recommended by the Agile framework.
    \item Learned about CI/CD processes to build and deploy images, a common software development practice in the industry when developing as part of a large team
    \item Gain experience in context switching between multiple paradigms, a common trait of software engineers working within development teams on large projects.
    \item Realised common differences and patterns across different languages and being able to leverage these differences to solve particular problems.
    \item Reinforced the importance of documentation, and separating one's experience when writing documentation to provide a better user experience.
\end{itemize}
\subsubsection{Non-Technical Learning Outcomes}
\begin{itemize}
    \item Experience was gained in dealing with Product Owners and Stakeholders, from listening to requirements and recommendations along with the confidence to push back with my own ideas.
    \item Learned the benefits of tools such as JIRA and Trello in planning work. Included here was learning how to properly track work progress and utilise this information to improve work output while reducing technical debt.
    \item Communication skills improved, from general day to day contact that is common in a development ecosystem, across mediums such as IRC, Email, Slack etc. Right up to gaining the confidence to present and project my ideas and thoughts to a larger audience. 
    \item Writing experience and skill improved dramatically over the course of this project. This includes academic writing, technical documents, pull requests and code reviews.
    \item The ability to retrospectively self evaluate oneself, and learn from mistakes. This is a key skill not only in software engineering but in any career.
\end{itemize}
\clearpage
\subsection{Personal Reflection}
This has been an amazing journey from start to finish, I feel it has helped me progress as a not only a developer but also a person, learning many new skills and discovering new abilities, mixed with making new connections and gaining exposure to new and exciting platforms. This project will help me proceeding forward with a career Software Engineering. 
\\The highlight to my project was the involvement of Red Hat and the Knative community, who assisted my problem-solving in several areas of this project. Overall, this demonstrates the importance and benefits a community can bring to a project. Open Source stretches far beyond a public repository on GitHub, and my FYP gave me an insight into what it takes to maintain such a project and how to successfully contribute to one. Having a contribution to the Knative project is a personal success for me, and something which I did not foresee when beginning my FYP.
\\Over the course of my time as an Applied Computing student, dealing with many projects and deadlines, along with my time as an Intern at Red Hat and StitcherAds, I have constantly been taught methods and processes on how to take an idea and see it to fruition. This project gave me to opportunity to take the lessons learned and put them into practice. It showed me the importance of initially understanding a problem. While the process of splitting a problem into small tasks can be quite time consuming and tedious, doing this at a granularity at which I never had to sole responsibility to do before proved to pay dividends as the project progressed. This exposed and opened new and previously unthought of avenues throughout the project. Overall this increased the project's value and when it came time to developing these tasks there was zero confusion allowing the completion of these tasks happen in a fluid and consistent manner.
\\The project followed the Scrum methodology; while not being targeted for a solo project I adopted the methodology to suit. Numerous times throughout this project I questioned was this the correct methodology to suit an FYP. Now retrospectively looking back I feel it was the correct choice, while my arguments against scrum revolved around it not being dynamic enough, as college circumstance change so quickly, it was often hard to adapt sprints to suit. While this is a valid argument, I feel the commitment of scrum is needed for an FYP, in that as college circumstances change you are obligated to finish all tasks in a current sprint. This ensures work gets completed regardless, and progress is consistently made. I feel being more dynamic would lead to less completed work with added pressure on the final deadline. That said this is a topic that could do with its own body of research. As it is such a broad area and the benefits of correct planning and management are apparent.
\\This project allowed me to explore multiple paradigms; this was one of my own personal goals, as I wanted to build my confidence as a software engineer. Unbeknownst to myself it opened my eyes to the differences across these paradigms and the value some can bring to particular problems and not others. While some languages possessed similarities, which I could leverage to solve the problem, others forced me to think of the problem in a different light and approach it differently. This plays back into the importance of planning, and giving the time to break problems and tasks into small manageable chunks, ensuring all aspects are clear and the task is feasible. While increasing the overall challenge to the project, this taught me valuable lessons, and the importance of having the right tool for the job.
\\Overall I feel my project was a success, and has motivated me to pursue a career in Open Source and the Kubernetes eco-system along with continuing work on RubiX.


\appendix
\section*{Appendices}
\addcontentsline{toc}{section}{Appendices}
\renewcommand{\thesubsection}{\Alph{subsection}}

\subsection{Moby Seccomp Defaults} 
\label{appendix:seccomp}
https://raw.githubusercontent.com/moby/moby/master/profiles/seccomp/default.json

\subsection{Docker Capabilities}
\label{appendix:capab}
\begin{figure}[!ht]
\centering
\includegraphics*[width=\textwidth]{images/term5.png}
\caption{\em Docker capabilities by default}
\end{figure}
\newpage
\subsection{CIS Guide Extract}
\label{appendix:cis}
\textbf{1.1 Ensure a separate partition for containers has been created (Scored)}

\textbf{Profile Applicability:}

Level 1 - Linux Host OS 

\textbf{Description:}

All Docker containers and their data and metadata is stored under /var/lib/docker directory. By default, /var/lib/docker would be mounted under / or /var partitions based on availability.

\textbf{Rationale:}

Docker depends on /var/lib/docker as the default directory where all Docker related files, including the images, are stored. This directory might fill up fast and soon Docker and the host could become unusable. So, it is advisable to create a separate partition (logical volume) for storing Docker files.

\textbf{Audit:}

At the Docker host execute the below command:

\textit{grep /var/lib/docker /etc/fstab}

This should return the partition details for /var/lib/docker mount point. 

\textbf{Remediation:}

For new installations, create a separate partition for /var/lib/docker mount point. For systems that were previously installed, use the Logical Volume Manager (LVM) to create partitions.

\textbf{Impact:}

None.

\textbf{Default Value:}

By default, /var/lib/docker would be mounted under / or /var partitions based on availability.

\textbf{References:}

https://www.projectatomic.io/docs/docker-storage-recommendation/

\textbf{CIS Controls:}

14 Controlled Access Based on the Need to Know Controlled Access Based on the Need to Know

\newpage
\bibliography{bibliography}

\end{document}