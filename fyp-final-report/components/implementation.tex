\section{Implementation}

\subsection{Sprint 0}
\subsubsection{Sprint Planning}
A number of stories where completed last semester as part of building a prototype for the project. Following the Scrum methodology, a backlog grooming took place in order to reevaluate the current backlog, eliminating stories which where complete and reordering stories based on new insight to the project, discovered during the development work on the prototype.
\\With initial completion of the JavaScript SDK, it was decided to begin work on the CLI. Laying the ground work on the CLI during this early sprint would add value to the project by paving the way for further development later on.
\subsubsection{Sprint Review}
The following goals where achieved:
\begin{itemize}
    \item CLI Repo was built and CI/CD configured.
    \item Cobra scaffolding was built for CLI
    \item Init Function was completed
    \item CLI Documentation was completed
    \item Restructure of JS SDK was completed
\end{itemize}
During the Sprint Review a Stake Holder highlighted that a user might want to change the name of their project after it has been initialized. This topic was discussed and a story created and added to the backlog.
\subsubsection{Sprint Retrospective}
\textbf{What did we do well?}
\begin{itemize}
    \item Upskilling in Golang
\end{itemize}
\textbf{What could have been done better?}
\begin{itemize}
    \item Better use of story points, as Sprint output was under estimated.
    \item Backlog refinement could of been improved
    \item User definitions could be clearer.
\end{itemize}
\textbf{Actions}
\begin{itemize}
    \item Ciaran Roche should revise current backlog stories
    \item Story Points needed at the Story level over the use of Story Points at the task level
    \item Proper usage of labels to help organise sprints and tasks via trello as well as JIRA
\end{itemize}

\subsubsection{Sprint Burndown}
\begin{figure}[!hb]
\centering
\includegraphics*[width=0.7\textwidth]{images/burndown-sprint-1.png}
\caption{\em Sprint 0 Burndown Chart}
\label{img:bd1}
\end{figure}

\subsubsection{Personal Reflection}
This sprint highlighted areas which I overlooked, mainly those mentioned in the \textit{Actions} from this Sprint Retrospective. The main take away from this sprint was the importance of Story Points, as it is a key factor in discovering sprint output and planning more efficient sprints in the future. I feel this is reflected nicely in Figure \ref{img:bd1}, which shows all tasks being completed ahead of sprint end. Overall I was happy with the outcome of this sprint as it was the first one there was bound to be teething problems.

\subsection{Sprint 1}
\subsubsection{Sprint Planning}
This sprint focused on two topics, the first being the CLI. It was decided to focus on adding capabilities to build and push an image to a repository. This feature would add the most value early on in development of the project. The second topic was to begin work on the showcase application. Taking this sprint to spike on a suitable environment to host the showcase along with begin work on appropriate documentation for the entire project. As the project is spread across multiple repositories a central location for documentation will bring value and usability to those looking to use the tool.
\subsubsection{Sprint Review}
The following goals where achieved:
\begin{itemize}
    \item Create functionality was added to CLI.
    \item Push functionality was added to CLI.
    \item Parsing of the RubiX schema was completed within the CLI.
    \item A Documentation Repo was set up.
    \item An environment was decided on, using Google Kubernetes Engine (GKE).
    \item Configuring and running a function within GKE was documented.
\end{itemize}
\subsubsection{Sprint Retrospective}
\textbf{What did we do well?}
\begin{itemize}
    \item Finishing the Sprint ahead of time
\end{itemize}
\textbf{What could have been done better?}
\begin{itemize}
    \item Better plan the Sprint to avoid early completion of Sprint
    \item Time management improvements as more work is consumed on the Weekends
\end{itemize}
\textbf{Actions}
\begin{itemize}
    \item Refine story points, as what was considered a 5 might not be a 5 as we begin the third sprint, stories which where initially pointed might not be as complex as previously thought, leading to sprint work finishing early.
    \item Look to improve time management to ensure a more consistent completion of work over large chunks at the weekend.
\end{itemize}
\clearpage
\subsubsection{Sprint Burndown}
\begin{figure}[!ht]
\centering
\includegraphics*[width=0.8
\textwidth]{images/sprint-2.png}
\caption{\em Sprint 1 Burndown Chart}
\label{img:bd2}
\end{figure}
\subsubsection{Personal Reflection}
Following the incremental improvements instilled in Scrum development, figure \ref{img:bd2} shows a slight improvement on the previous sprint. While being an improvement I feel that there is quite a lot more room to further better my sprints. The burndown chart shows work was added to sprint due to over all sprint velocity being underestimated. The action points from this sprint reflect this with a full revision of story points needed to reevaluate the complexity of stories.
\\All that said, while it is still early days within the development process, I am beginning to think, was scrum the right approach for a final year project. My argument being, in college workloads change week on week, with focus on other assignments and modules. Would a methodology which is more reactive allowing for the flexibility to add and remove work at will from sprints, have being a better approach? \\ This is a question I hope the answer by the end of this semester, as it is still early and it takes time to figure out a teams skill level and velocity. In my case being a one person team the factors that affect velocity are different again. Overall I am quite happy with my progress as I feel in a stronger position being able to complete sprints early as opposed to not completing sprints. It adds to my excitement for the project to watch these changes sprint on sprint and see my project and my processes improve.

\subsection{Sprint 2}
\subsubsection{Sprint Planning}
A lot of time was given to refining the backlog and giving priority to stories that will add the greatest amount of value at this midway sprint.
\\With that, this sprint took the opportunity to return to the JavaScript SDK and finish it to a production quality. As the work completed to date on the SDK was work carried out during the proof of concept phase. Included in this work was ensuring the SDK fails safely with appropriate error handling. Along with a context parser to allow for metadata to be written to request headers.
\\Another story brought into this sprint looked to develop an SDK to support GoLang. This SDK would need to support all the features of the JavaScript SDK.
\\Two further stories where brought into the sprint to design the Showcase Application.
\subsubsection{Sprint Review}
The following goals where achieved:
\begin{itemize}
    \item Complete work on JavaScript SDK
    \item Complete work on GoLang SDK
    \item Develop GoLang example app
\end{itemize}
\subsubsection{Sprint Retrospective}
\textbf{Note :} At the beginning of this sprint I was bed bound after coming down with flu. Unable to begin work sprint work till the second week of the sprint.
\textbf{What did we do well?}
\begin{itemize}
    \item Complete a vast amount of the allocated work in the final days of the sprint
    \item The reorder of the backlog during the planning
\end{itemize}
\textbf{What could have been done better?}
Due to the sickness, not enough working time was consumed over the two week period to give us enough information on areas that could be improved.
\\\textbf{Actions}
\begin{itemize}
    \item Take a RAID approach to planning.
\end{itemize}
The acronym RAID stands for Risks, Assumptions, Issues and Dependencies. This covers events that will have an adverse impact on the project if they occur. In this case working on a one person team my inability to work on the project during a sprint had an impact on output, along with giving no valuable information that could be used to further improve work output sprint on sprint.
\subsubsection{Personal Reflection}
Personally this was a disappointing sprint, in that I was unable to complete work due to being sick. When you begin a sprint you are committing to completing all work allocated to the sprint, not being able to complete this goal felt like a set back, as the previous sprints saw a higher output then what was originally planned.
\\That said it was encouraging to see the amount of work I was able to complete in a short space of time. I originally thought it would be an easy task to port the SDK from one language to another. As I began this work on the GoLang SDK I immediately saw this was not as straight forward as I predicted. Despite the original SDK being wrote in TypeScript which is strictly typed, it still allowed me some flexibility in my code. Writing the SDK in GoLang did not offer me any freedom, and with this, decisions needed to be made to insure the GoLang SDK performed as expected. This was a challenge as time was against me, but was also an enjoyable experience while I improved my efficiency with working with GoLang.
\\I feel the need to return to the question that came up in the previous decision, is Scrum the correct methodology a final year project should follow? While this question could be the topic of an entire research project, I feel after completing this sprint with an unforeseen event affecting the output of the work. I do think a more reactive approach to the methodology would allow a lot more freedom while alleviating some pressure. That said, the commitment to complete work outlined in the sprint, was a driving force behind the development that was completed. Having freedom here, would the work have being finished or would it have been pushed back? I cannot make up my mind, with two more sprints to go before this project is over will the benefits of working in a one person Scrum team become clear? Time will tell.
\subsubsection{Burndown}
\begin{figure}[!ht]
\centering
\includegraphics*[width=0.8
\textwidth]{images/sprint-3.png}
\caption{\em Sprint 2 Burndown Chart}
\label{img:bd3}
\end{figure}

\subsection{Sprint 3}
\subsubsection{Sprint Planning}
As with all sprint planning, choosing the stories to complete in the coming sprint is based on priority along with what will add the greatest value to the MVP at the end of this project. Looking back at the work completed so far, with the CLI in good shape, and progress being made toward adding SDK's to the frame work it was decided to further the frame works portfolio of SDK's. Therefore, stories for this sprint included the completion of both a Java and a Python. Along with stories which would allow for the bootstrapping and building of the new SDK's through the CLI.
\subsubsection{Sprint Review}
The following goals where achieved:
\begin{itemize}
    \item Complete work on Python SDK
    \item Add Python functions to CLI
    \item Complete work on Java SDK
    \item Restructure GitHub Org and Documentation
    \item Complete planning work of Showcase Application
\end{itemize}
The following goals were not achieved:
\begin{itemize}
    \item Add Java functions to CLI
\end{itemize}
\subsubsection{Sprint Retrospective}
\textbf{What did we do well?}
\begin{itemize}
    \item Context switching between various languages.
    \item Sprint Planning is getting stronger and more accurate. 
\end{itemize}
\textbf{What could have been done better?}
\begin{itemize}
    \item Reach out for assistance quicker.
\end{itemize}
\textbf{Actions}
\begin{itemize}
    \item Maintain channel of communication between developer and stakeholders throughout sprint.
\end{itemize}
\subsubsection{Personal Reflection}
Before delving into my personal reflection it is worth pointing out and adding context to a number of the above points made in the Retrospective. First, despite this being the first sprint not to achieve all the goals, the sprint planning is becoming more accurate as a flow for the sprint output is becoming more apparent and the value of story points becomes more refined. The reasoning behind not finishing all tasks is due to the nature of dealing with a strongly typed language such as Java, the planning for the SDK needed to be adapted to suit. While I waited to long to reach out for advice on how to overcome the problem at hand. When a solution was found I was left unable to deploy the SDK to Maven before the sprint end. Without a release of the SDK the work for bootstrapping and the build functions could not be complete.
\\With that said while the work on the Java SDK was disappointing on my behalf, but the overall sprint was enjoyable with lots of lessons learned. The most important lesson is choosing the right time to ask for help. I feel I left this too late in the sprint to be able to complete the sprint goal. Working on an FYP can be quite an isolated experience due to the work being carried out, is graded on the premise that you completed it yourself, along with the fact you are essentially a one person team. While in the 'real world' this type of environment would be extremely rare due to the emphasis on team work. I have to remind myself that while this is an FYP I have a team around me, to guide me and advice me. This team consists of my supervisor, the stakeholders and my lecturers.
\\Another important take away from this sprint revolves around context switching between paradigms. The initial SDK work was completed in TypeScript, while it requires types to be declared it still offers some of the freedom of dynamic languages. GoLang was the next SDK to be completed, which this is strongly typed it feels quite dynamic. It allows a certain amount of freedom, and this freedom lead to an easy transition from TypeScript to Golang. Python falls into this category, being an easy transition. I guess I got too comfortable working with these languages, which lead to me overlooking the complexity of creating a Java SDK. This is something I will be taking forward with me, to not make assumptions and where possible to ask questions and get verification.
\\With lessons learned and work progressing this was a challenging sprint. Leaving me eager to continue work and watch my own progression with juggling FYP, general college work and personal life. With less then half the semester left, the excitement and the nerves of completing this project are at an all time high. 
\subsection{Sprint 4}
\subsubsection{Sprint Planning}
Work on developing the showcase application took precedence this Sprint. With two weeks given to create the showcase outlined in Section \ref{sub:showap}. This would allow the opportunity to see how the RubiX framework handles working on a real world project. Along with building more familiarity with Knative and Kubernetes.
\subsubsection{Sprint Review}
The following goals where achieved:
\begin{itemize}
    \item Complete CRUD functions with RubiX
    \item Develop a function that incorporates multiple API's with RubiX
    \item Develop frontend Application to consume RubiX functions
    \item Deploy Application and Functions to GKE
\end{itemize}
\subsubsection{Sprint Retrospective}
\textbf{What did we do well?}
\begin{itemize}
    \item Collaboration with the Knative Community
    \item Gaining familiarity with the Google Cloud Platform
\end{itemize}
\textbf{What could have been done better?}
\begin{itemize}
    \item More investigation into the tools I am using
\end{itemize}
\textbf{Actions}
\begin{itemize}
    \item Maintain collaboration with the Knative Community
\end{itemize}
\subsubsection{Personal Reflection}
Looking at the burndown chart in Figure \ref{img:bd4} it shows a steady progress of completion of tasks. While the progress was steady and consistent it was not an easy road. It required reaching out to the Knative community, several slack chats with Knative engineers, a GitHub issue and multiple google group threads later the app was deployed. 
\\It was a good learning experience and showed the power of open source, having the resources to be able to reach out to those working and developing the software. As Knative is so new, without a full release and with consistent change some of what I wanted to achieve was yet to be documented and in one case has not been developed yet. One instance was Knative does not support CORS configuration at the Knative Ingress Gateway, resulting in access to services in a Knative cluster to be impossible. It was a Knative engineer who suggested using xip.io to create a new namespace for my cluster, and through this I was able to bypass the need for CORS config at the Ingress Gateway. A simple yet clever workaround for an undeveloped feature.
\\Away from that I was personally happy with the progression of this sprint, showing my planning had improved and my sprint consumption had remained consistent. This was the first sprint in which I brought the framework beyond basic use cases, and I was quite happy with how the framework worked and handled. Which left me with a reassurance of the value to which this framework brings.
\subsubsection{Burndown}
\begin{figure}[!ht]
\centering
\includegraphics*[width=0.8
\textwidth]{images/sprint-4.png}
\caption{\em Sprint 4 Burndown Chart}
\label{img:bd4}
\end{figure}
\subsection{Sprint 5}
\subsubsection{Sprint Planning}
It was decided to make this the final sprint of the Project. So the tasks included creating the first MVP release of the Project. This included tasks of increasing test coverage in critical repos, documentation verification, and manual testing of all repos. A story was included in this sprint to develop a new SDK to support a functional programming paradigm (Haskell).
\subsubsection{Sprint Review}
The following goals where achieved:
\begin{itemize}
    \item Increase test coverage to 90\% in the CLI and JS SDK Repos
    \item Verified all documentation
    \item Completed v1 release of Framework
    \item Developed Haskell SDK
\end{itemize}
\subsubsection{Sprint Retrospective}
\textbf{What did we do well?}
\begin{itemize}
    \item Complete all MVP tasks and release v1
    \item Haskell SDK 
\end{itemize}
\subsubsection{Personal Reflection}
This was quite an emotional sprint due to being the final sprint for this current project. It gave me two weeks to step back and look back at the work completed over all sprints as I performed manual testing and verification. 
\\The only down side to this sprint was in that, JIRA updated their sign in which left me locked out of JIRA, so that I could not get a burndown chart for my final sprint, while disappointing not to have the chart, it was not a burden or a set back as all planning work was completed in Trello it was quite an easy transition to track my tasks there. 
\\Taking on the work of supporting Haskell in the framework, was one of the hardest challenges of the whole project. The differences between functional programming in comparison to dynamic languages or static, meant I could not leverage the existing patterns of other SDK's, I had to approach the Haskell SDK from a different angle. While challenging it was quite a learning experience, and I feel it lead to the development of a strong product.
\\A number of weeks back at the Waterford Tech Meet Up, Richard Rodgers spoke about the development of his latest startup, when he got to the topic on testing, he showed the variation of test coverage across his different services, ranging from zero percent to plus ninety percent in some cases. He argued that at the early stages of development of a new product, when there are goals to be met, in creating an MVP having full test coverage does not always work, and priority has to be given to core services and development of features. I took this mentality with my project as I strived for breath over depth, trying to gain as much experience and exposure through a single project as I could. This sprint gave me a sense that I completed this goal, as I look across the RubiX org on GitHub and see the variations of different paradigms, and the individual lessons and challenges each one possessed.

