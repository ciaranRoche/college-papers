\section{Summary}
What can I say? If I am honest this practical was not a nice dev experience. The first and it being the least relevant to SDN and Openflow as a technology, was my own lack of knowledge in Python. Small little nuances where tripping me up in this practical and meant more time was spent upskilling in python trying to get a grasp on the paradigm before I could tackle Openflow.
\\Away from that, I found the documentation poor, and also hard to find. Googling Openflow lead to an abundance of blog posts on the protocol and nothing on official documentation, which when found was poor. Mixed with my lack of python knowledge lead to a poor dev experience. This meant I leaned toward, going to blog posts and github for help, but due to different naming conventions between the versions often meant time was spent doing something that was incompatible with my version.
\\At face value my solution outlined in the report appears quite simple, but in reality to get to that simple solution took quite an amount of time to achieve. This leads me to think a greater level of abstraction is needed to provide a dev friendly experience. As this was a very basic example, I could see that it would become quite complex quite quickly depending on the use case. 
\\There was also a couple of weird behaviors which I noted, when dumping the flows from the switch sometimes would not reflect the order in which they where inserted. This lead to some head scratching as I am used to order being pedantic within ACL's etc. On the topic of the likes of ACL's, this may be due to the example topology I used, I kept asking myself what value is Openflow adding, in that I am creating OpenFlow traffic on top of the regular network traffic and basically achieving the same functionality as the network would have without Openflow. That said looking at the bigger picture having a central control over the network would add a lot of value. 
\\To sum up, a view of Openflow in a bigger environment is needed to gain a greater appreciation of the protocol. Also a greater level of abstraction to provide a cleaner developer experience.