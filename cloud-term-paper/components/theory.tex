\section{CoreOS Operators}
\label{sub:theory}
From the example of an Operator working in Section \ref{sub:etcd} we can conclude that an Operator is a process for packaging, deploying and managing Kubernetes applications. To define what a Kubernetes application is is an application that is deployed on Kubernetes but also managed using the Kubernetes APIs and the Kubectl tooling \citep{coreosblog}.

Typically a Site Reliability Engineer (SRE) is one who manages an application by writing software. And also has the know how to develop software for a specific application domain. This piece of software now has operational domain knowledge built into it. Taken from this perspective we can look at Operators as an implementation of this concept, a piece of software that can create, configure and manage Kubernetes applications.

It does this by extension of the Kubernetes API and builds atop of the basic Kubernetes resource and controller concepts, but infuses domain and or application-specific knowledge to automate common tasks \citep{coreosintro}

\subsection{Operator Framework}
In order to make the development of Operators and the management of Kubernetes applications easier, Red Hat and the Kubernentes open source community created the Operator Framework. The framework is a toolkit designed to manage Kubernetes native applications in an automated and elastic way \citep{framework}. The following sections looks at some of the tool kits within the Operator Framework.

\subsection{Operator SDK}
Creating Operators can be difficult due to the complexities of using low level APIs, writing your own boilerplace and the overall lack of modularity which often leads to duplication. With this in mind the Operator Framework added an Operator SDK toolkit. The SDK provides the tools to build, test and package Operators. The best practises and patterns are included in the SDK to help prevent reinventing the wheel, Figure \ref{img:op_sdk} illustrates the development cycle for working with the SDK \citep{sdk}.

Typically Operators are developed in Go but can also be developed in Ansible the following outlines the workflow for a Go Operator:
\begin{itemize}
    \item Use the SDK CLI to create a new operator project
    \item Define new resource APIs through a Custom Resource Definition (CRD)
    \item Define Controllers to watch resources
    \item Write reconciling logic for the Controller by utilizing the SDK and controller-runtime APIs
    \item Use the SDK CLI to build and generate the operator deployment manifest
\end{itemize}
This workflow will be demonstrated in Section \ref{sub:practical}.

\begin{figure}[!ht]
\centering
\includegraphics*[width=0.7\textwidth]{images/op20.png}
\caption{\em Operator SDK workflow (Image from CoreOS)}
\label{img:op_sdk}
\end{figure}

\subsection{Operator Lifecycle Manager}
Built operators need to be deployed on a Kubernetes cluster. It is the Operator Lifecycle Manager (OLM) facilitates the management of Operators on a Kubernetes cluster. True the Lifecycle Manager administrators are able to control what Operators are available in what namespaces and who is able to interact with them. It also gives the capabilities to trigger updates to resources and the Operator itself \citep{operator-framework_2018}.
\\ To put differently if we think of an Operator of a mechanism to manage our pods, how do we manage our Operator, does it require its own Operator? Well the Operator Lifecycle Manager extends Kubernetes to provide a declarative way to install, manage and upgrade operators and their dependencies within a cluster. As of writing this report the OLM currently supports the definition of applications as a single Kubernetes resource and encapsulates these requirements and the metadata. It allows for the install of applications automatically with dependency resolution, and finally it can upgrade applications automatically with different approval policies.

\begin{figure}[!ht]
\centering
\includegraphics*[width=0.7\textwidth]{images/op21.png}
\caption{\em Operator SDK Lifecycle Manager (Image from CoreOS)}
\label{img:op_lifecycle}
\end{figure}

\subsection{Operator Metering}
As of writing this report, Operator Metering is not released yet and plans to be released in the coming months.\\ It will enable usage reporting for Operators that provide specialized services. It will do this through the recording of historical cluster usage, from these records it will generate reports showing usage breakdowns by pod or namespace over time periods. The overall project aim is to tie into the cluster CPU and memory reporting, this will allow for a calculation of Infrastructure as a Service costs \citep{operator-metering_2018}.