\section{Summary}

\subsection{Project Direction}
Currently the project has succeeded in delivering an MVP to the Product Owners. With the completion of the MVP I feel the project has successfully solved the problem outlined in Section \ref{sub:problem}, in providing a means to quickly and efficiently develop applications within Knative. The current state of the Project gives users an open door to explore what the Knative project has to offer, with minimal upskilling needed, giving more focus to Knative and its features.
\subsubsection{Future Development}
I feel the project currently solves the initial outlined problem, while there is work to be completed, which is documented and logged on RubiX backlog in Trello. With exposure and experience to the Kubernetes and Knative ecosystem I have discovered many new demands and needs for developers, this is further backed by the release of Quarkus\footnote{https://quarkus.io -- A Kubernetes Native Java stack.}. A high level view of Quarkus, is a container-centric Java Stack with the main goal to spin up a Java container in sub second times. 
\\The current state of RubiX provides the ground work to develop upon. Work that continues will solely focus on a single SDK, with the core goal of container first, developing near instant scale up and high density memory utilization in Kubernetes. As current trends in development within the Knative project are skewed toward scaling from 0 in sub second times. The container itself is an individual cog in this process, and development in this area is extremely valuable to the Knative community. RubiX is the perfect candidate to build upon, and this is work which I will begin on after the college term.
\subsection{Review}
\subsubsection{Core Goals}
The project delivered on all its core goals set out by the stakeholders.
\begin{itemize}
    \item [\textbf{AO1}] - Completed SDK's in the following paradigms -- TypeScript, GoLang, Python, Java and Haskell
    \item [\textbf{AO2}] - Each SDK handling and catching errors, with the ability for logging.
    \item [\textbf{AO3}] - Dockerfiles following the best practices outlined by Docker and the OCI.
    \item [\textbf{AO4}] - The project was developed in the open under the GitHub org RubixFunctions.
    \item [\textbf{AO5}] - A CLI tool developed to abstract common usages and workflows through the development of Functions as a Container.
\end{itemize}
The project deliverables are as follows:
\begin{itemize}
    \item JavaScript SDK
    \item GoLang SDK
    \item Python SDK
    \item Java SDK
    \item Haskell SDK
    \item Development CLI
    \item Knative Terraform
    \item RubiX/Knative Showcase Application
    \item RubiX Documentation
    \item RubiX HelloWorld Samples
\end{itemize}
\textbf{Contributions}
\\As Knative is an Open Source project, it gave the opportunity to contribute back to it over the course of the project. It is often a misconception that a contribution to an Open Source project has to be in the terms of code. But a contribution comes in many forms, with the goal of improving the project and developing something special and community driven.
That said over the course of my Project it resulted in the following:
\begin{itemize}
    \item One Pull Request to a Knative Repository.
    \item One Github Issue in a Knative Repository.
    \item Two authored Knative Google Group Threads.
    \item Responded to multiple Knative Google Group Threads.
    \item Partook in Slack Discussions.
\end{itemize}

\newpage
\subsection{Learning Outcomes}
This project produced multiple Learning Outcomes, ranging from Technical and Non-Technical.
\subsubsection{Technical Learning Outcomes}
\begin{itemize}
    \item Gained valuable insight into the containerization and orchestration of applications
    \item Developed an appreciation for quick prototyping and iterating within the development process, as is recommended by the Agile framework.
    \item Learned about CI/CD processes to build and deploy images, a common software development practice in the industry when developing as part of a large team
    \item Gain experience in context switching between multiple paradigms, a common trait of software engineers working within development teams on large projects.
    \item Realised common differences and patterns across different languages and being able to leverage these differences to solve particular problems.
    \item Reinforced the importance of documentation, and separating one's experience when writing documentation to provide a better user experience.
\end{itemize}
\subsubsection{Non-Technical Learning Outcomes}
\begin{itemize}
    \item Experience was gained in dealing with Product Owners and Stakeholders, from listening to requirements and recommendations along with the confidence to push back with my own ideas.
    \item Learned the benefits of tools such as JIRA and Trello in planning work. Included here was learning how to properly track work progress and utilise this information to improve work output while reducing technical debt.
    \item Communication skills improved, from general day to day contact that is common in a development ecosystem, across mediums such as IRC, Email, Slack etc. Right up to gaining the confidence to present and project my ideas and thoughts to a larger audience. 
    \item Writing experience and skill improved dramatically over the course of this project. This includes academic writing, technical documents, pull requests and code reviews.
    \item The ability to retrospectively self evaluate oneself, and learn from mistakes. This is a key skill not only in software engineering but in any career.
\end{itemize}
\clearpage
\subsection{Personal Reflection}
This has been an amazing journey from start to finish, I feel it has helped me progress as a not only a developer but also a person, learning many new skills and discovering new abilities, mixed with making new connections and gaining exposure to new and exciting platforms. This project will help me proceeding forward with a career Software Engineering. 
\\The highlight to my project was the involvement of Red Hat and the Knative community, who assisted my problem-solving in several areas of this project. Overall, this demonstrates the importance and benefits a community can bring to a project. Open Source stretches far beyond a public repository on GitHub, and my FYP gave me an insight into what it takes to maintain such a project and how to successfully contribute to one. Having a contribution to the Knative project is a personal success for me, and something which I did not foresee when beginning my FYP.
\\Over the course of my time as an Applied Computing student, dealing with many projects and deadlines, along with my time as an Intern at Red Hat and StitcherAds, I have constantly been taught methods and processes on how to take an idea and see it to fruition. This project gave me to opportunity to take the lessons learned and put them into practice. It showed me the importance of initially understanding a problem. While the process of splitting a problem into small tasks can be quite time consuming and tedious, doing this at a granularity at which I never had to sole responsibility to do before proved to pay dividends as the project progressed. This exposed and opened new and previously unthought of avenues throughout the project. Overall this increased the project's value and when it came time to developing these tasks there was zero confusion allowing the completion of these tasks happen in a fluid and consistent manner.
\\The project followed the Scrum methodology; while not being targeted for a solo project I adopted the methodology to suit. Numerous times throughout this project I questioned was this the correct methodology to suit an FYP. Now retrospectively looking back I feel it was the correct choice, while my arguments against scrum revolved around it not being dynamic enough, as college circumstance change so quickly, it was often hard to adapt sprints to suit. While this is a valid argument, I feel the commitment of scrum is needed for an FYP, in that as college circumstances change you are obligated to finish all tasks in a current sprint. This ensures work gets completed regardless, and progress is consistently made. I feel being more dynamic would lead to less completed work with added pressure on the final deadline. That said this is a topic that could do with its own body of research. As it is such a broad area and the benefits of correct planning and management are apparent.
\\This project allowed me to explore multiple paradigms; this was one of my own personal goals, as I wanted to build my confidence as a software engineer. Unbeknownst to myself it opened my eyes to the differences across these paradigms and the value some can bring to particular problems and not others. While some languages possessed similarities, which I could leverage to solve the problem, others forced me to think of the problem in a different light and approach it differently. This plays back into the importance of planning, and giving the time to break problems and tasks into small manageable chunks, ensuring all aspects are clear and the task is feasible. While increasing the overall challenge to the project, this taught me valuable lessons, and the importance of having the right tool for the job.
\\Overall I feel my project was a success, and has motivated me to pursue a career in Open Source and the Kubernetes eco-system along with continuing work on RubiX.
