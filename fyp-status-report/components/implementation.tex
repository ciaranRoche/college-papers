\section{Implementation of Prototype}
\label{sub:poc}
To ensure full technical feasibility, two Prototypes were developed. The first was used as a benchmark for further iterations, ensuring that work carried out within this project not only improves the user experience, but also improves on the existing solution technically. The second prototype was developed following the design laid out in Section \ref{sub:design}.

\subsection{Prototype 1}
\label{sub:proto1}
As outlined in Section \ref{sub:aims}, essentially a Function as a Container is a function wrapped in a server and packaged as an image. Through the development of this prototype a server was built in Typescript. Included in this server were two endpoints that satisfy a HTTP Get Request and a HTTP Post Request. This was built with the aid of a common Node Module \textit{Express}\footnote{https://expressjs.com -- Minimalist web framework for Node.js} and followed recommended patterns of Middlewares / pipelines. This is a simple abstraction in that the output of one unit/function is the input for the next
\\This prototype serves as a benchmark for further implementations as the project looks to provide a better solution to the development of a Function as a Container. It provides a baseline of a typical Container for us to compare. The image for this prototype can be found at Appendix \ref{appendix:proto1}.

\subsection{Prototype 2}
\label{sub:proto2}
The resulting code from this Prototype of the SDK can be found at Appendix \ref{appendix:proto2}, along with a showcase application built to demo it, which can be found at Appendix \ref{appendix:sdkshowcase}. Completed in this Prototype is the ground work for the SDK, written in Typescript allowing a user to create a Function as a Container. This provided the feasibility for the project and the design outlined throughout this paper.

\subsection{Prototype Summary}
\label{sub:protosum}

\begin{table}[!ht]
\centering
\begin{tabular}{cccc}
\hline
\multicolumn{1}{l}{Prototype} & \multicolumn{1}{l}{Size} & \multicolumn{1}{l}{Vulnerabilities-} & \multicolumn{1}{l}{Latency} \\ \hline
One & 75.3MB & 90 & 15ms \\
Two & 23.4MB & 0 & 7ms \\ \hline
\end{tabular}
\caption{Prototype Comparison Table}
\label{table-comp}
\end{table}
A number of arbitrary statistics where taken from the two prototypes based on freely available statistics from \gls{Quay}\footnote{https://quay.io -- Container Repository}. These results can be seen in Table \ref{table-comp}. As we can see our prototype following the design structure outlined in this report has made improvements in both size and latency, while reducing the number of vulnerabilities within our \gls{Container}. This in my opinion is due to stripping away third party modules and dependencies, as it allows us to focus on what we need with out accumulating a large dependency tree which we do not control. This prototype will now be focused on for the remainder of this project.
